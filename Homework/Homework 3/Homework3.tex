\documentclass[addpoints]{exam}
\usepackage[legalpaper]{geometry}
\usepackage{etoolbox}
\usepackage[utf8]{inputenc}
\usepackage{amsmath, amsfonts}
\usepackage{enumerate}
\usepackage{hyperref}
\usepackage{relsize}
\newcommand{\sech}{\operatorname{sech}}
\newcommand{\R}{\mathbb{R}}
\newcommand{\Z}{\mathbb{Z}}
\newcommand{\lt}{\left}
\newcommand{\rt}{\right}
\newcommand{\pd}[2]{\frac{\partial #1}{\partial #2}}
\newcommand{\pdd}[3][]{\frac{\partial^{#1} #2}{\partial #3^{#1}}}
\newcommand{\lam}{\lambda}
\newcommand{\ansSpace}{\vspace{70mm}}


\title{Math 327 Homework 3}
\author{Sathvik Chinta}
\date{October 21th, 2022}

\begin{document}

\maketitle

\begin{questions}
\question For each of the following two sets, find the maximum, minimum, infimum, and
supremum if they are defined. Justify your conclusions. 

a. $\{ 1/n \text{ } | \text{ } n \text{ in } \mathbb{N} \}$ \\
maximum:

The set $\{ 1/n \text{ } | \text{ } n \text{ in } \mathbb{N} \}$ consists of values

\[\{1, \frac{1}{2}, \frac{1}{3}, \frac{1}{4}, ..., \frac{1}{n}\}\]

Notice that $1 > \frac{1}{2} > \frac{1}{3} > \frac{1}{4}, ..., \frac{1}{n}$. Thus, 
this set is strictly decreasing. Therefore, the maximum of this set is $1$.

minimum:

There is no minimum for this set. This is because the infimum is not a member of the set.

infimum:

For any $\frac{1}{n} \in \mathbb{N}$, $0 < \frac{1}{n}$. If there is an $\epsilon > 0$,
we can find an $n \in \mathbb{N}$ such that $0 < \frac{1}{n} < \epsilon$ by the 
Archimedean principle. Thus, the infimum of this set is $0$.

supremum:

The defintion of the supremum of a set is the least upper bound of the set. In other words, 
if there is an upper bound $b$ for the set, $b \geq x$ for all $x$ in the set. Since the 
maximum of this set is $1$, the supremum of this set is also $1$. Any value less than 
$1$ is not an upper bound for the set since $1$ is in the set, while any value greater
than $1$ is not the least upper bound for the set since $1$ is a smaller upper bound.

b. $\{ x \text{ in } \mathbb{R} \text{ } | \text{ } x^2 < 2 \}$ \\
maximum:

There is no maximum for this set since the supremum is not a member of the set.

minimum:

There is no minimum for this set since the infimum is not a member of the set.

infimum:

The infimum of this set is $-\sqrt{2}$. This is because $-\sqrt{2} \leq x$ for
all $x$ in the set. Any value greater than $-\sqrt{2}$ (which we denote as $x$)
is not a lower bound for the set because there must exist some real number $q$ 
in the set such that $-\sqrt{2} < q < x$ for all x. 

supremum:

The supremum of this set is $\sqrt{2}$. This is because $\sqrt{2} \geq x$ for all
$x$ in the set. Any value less than $\sqrt{2}$ (which we denote as $x$) is 
not an upper bound for the set because there must exist some real number $q$ 
in the set such that $x < q < \sqrt{2}$ for all x. 

\question 

a. Prove that if $n$ is a natural number, then $2^n >n$. \\

We prove by induction. The base case is $n = 1$. Then $2^1 = 2 > 1$. Thus, the base case is true.

Now, assume that $2^k > k$ for some $k \in \mathbb{N} < n$. We must show that $2^{k+1} > k+1$.

\begin{align*}
2^{k} &> k \\
\end{align*}

\text{multiply both sides by 2} \\

\begin{align*}
    2 * 2^{k} &>  2* k \\
    2^{k+1} &> 2 * k
\end{align*}

Now, we prove that $2 * k > k + 1$. We prove by induction. The base case is $k = 1$. Then $2 * 1 = 2 > 1 + 1$. Thus, the base case is true.

Now, assume that $2 * g > g + 1$ for all $g \in \mathbb{N} < k$. We must show that 
$2 * (g+1) > (g+1) + 1$. This is equivalent to showing that $2g + 2 > g + 2$.

\begin{align*}
2 * g &> g + 1 \\
\end{align*}

\text{add 2 to both sides} \\

\begin{align*}
    2 * g + 2 &> g + 3 \\
\end{align*}

Since $g + 3 > g + 2$ for all $g \in \mathbb{N}$, we have that $2g + 2 > g + 2$ for all $g \in \mathbb{N}$.

Thus, $2*k > k + 1$. Since $2 * k > k + 1$ for all $k \in \mathbb{N}$, we have that $2^{k+1} > 2 * (k + 1)$ for all $k \in \mathbb{N}$.

Thus, $2^n > n$ for all $n \in \mathbb{N}$.

b. Prove that $n$ is a natural number, then 
\[n = 2^{k_0}l_0\]
for some odd natural number $l_0$ and some nonnegative integer $k_0$. (Hint: if $n$ is odd, 
let $k= 0$ and $l = n$; if $n$ is even, let $A = \{k \text{ in } \mathbb{N} \text{ } | \text{ }
n = 2^{k}l \text{ for some } $l$ \text{ in } \mathbb{N}.\}$ By (a), $A \subseteq \{1, 2, ..., n\}$. Choose
$k_0$ to be the maximum of $A$.)\\

We prove by induction. The base case is $n = 1$. Then $1 = 2^{0}1$. Thus, the base case is true.

We assume that this is true for some $g$ in $\mathbb{N} < n$. Thus, $g = 2^{k_0}l_0$ for some
odd natural number $l_0$ and some nonnegative integer $k_0$. We must show that this is true for $g+1$
as well. 

If $g + 1$ is odd, then 

\begin{align*}
    g + 1 &= 2^0(g+1) \\
\end{align*}

Since $0$ is a nonnegative integer and $g+1$ is odd. 

If $g + 1$ is even, then $\frac{g + 1}{2}$ is some integer as well. By our 
hypothesis, $\frac{g + 1}{2} = 2^{k_0}l_0$ for some odd natural number $l_0$ and some nonnegative integer $k_0$. Thus,

\begin{align*}
    \frac{g + 1}{2} &= 2^{k_0}l_0 \\
    g + 1 &= 2^{k_0+1}l_0 \\
\end{align*}

Since $k_0 + 1$ is a nonnegative integer and $l_0$ is an odd natural number,
the statement is true for $g+1$ as well.

Thus, $n = 2^{k_0}l_0$ for some odd natural number $l_0$ and some nonnegative integer $k_0$ for all 
$n \in \mathbb{N}$.

\question A real number of the form $m/2^n$ where $m$ and $n$ are integers, is
called a dyadic rational. Prove that the set of dyadic rationals is dense in $\mathbb{R}$.\\

A set is dense in $\mathbb{R}$ if for every $a < b$, there is some $x$ in the target set
such that $a < x < b$. So, if we prove that $a < \frac{m}{2^n} < b$, then we have shown that
the set of dyadic rationals is dense in $\mathbb{R}$.

By Archimedes Principle we know that there must exist some $\frac{1}{n}$ in the interval 
$a < b$. So, we can re-write this as $\frac{1}{n} < b - a$.  By 2(a), we know that $2^n > n$, 
so the inequaltiy $\frac{1}{2^n} < \frac{1}{n} < b - a$ must hold. We claim that there must be 
some $m$ such that $a < \frac{m}{2^n} <b$. We prove this by contradiction. 

Assume there is no $m$ such that $a < \frac{m}{2^n} <b$. This will only be possible if 
the distance between the two numbers is less than $\frac{1}{2^n}$. In other words, 
$b - a \leq \frac{1}{2^n}$. However, this is a contradiction to what we have proven earlier 
since $\frac{1}{2^n} < \frac{1}{n} < b - a$. Thus, there must be some $m$ such that $a < \frac{m}{2^n} <b$.

Thus, the set of dyadic rationals is dense in $\mathbb{R}$.

\question For each of the following statments, determine whether it is true or false and 
justify your answer

(a) The set $\mathbb{Z}$ of integers is dense in $\mathbb{R}$

A set is dense in $\mathbb{R}$ if for every $a < b$, there is some $x$ in the target set
such that $a < x < b$. However, if we let $b - a \leq 1$, then there cannot be any $x \in \mathbb{Z}$
in this interval. For instance, if $b = 1$ and $a = 0$, then there is no $x \in \mathbb{Z}$ between
0 and 1. Thus, the set $\mathbb{Z}$ is not dense in $\mathbb{R}$.

(b) The set of positive real numbers is dense in $\mathbb{R}$

A set is dense in $\mathbb{R}$ if for every $a < b$, there is some $x$ in the target set
such that $a < x < b$. However, if we choose $b$ and $a$ to both be negative, then 
there cannot be any $x$ in the set of positive real numbers in this interval. For instance, if
$b = -1$ and $a = -2$, then there is no $x \in \mathbb{R}^+$ between -2 and -1. Thus, the set of
positive real numbers is not dense in $\mathbb{R}$.

(c) The set of $\mathbb{Q}/\mathbb{N}$ of rational numbers that are not integers is dense in $\mathbb{R}$

A set is dense in $\mathbb{R}$ if for every $a < b$, there is some $x$ in the target set
such that $a < x < b$. This set is dense in $\mathbb{R}$ because, by definition, $Q$ and $N$ are both 
dense in $\mathbb{R}$. Thus, the set of $\mathbb{Q}/\mathbb{N}$ is dense in $\mathbb{R}$. 

\question Suppose that the number $a$ has the property that for every natural number $n$, 
$a \leq 1/n$. Prove that $a \leq 0$. 

We prove by contradiction. Let $a > 0$. Let $\epsilon$ be some positive real number. 
Then, there must exist some $n$ such that $0 < \frac{1}{n} < \epsilon$ for all $n$. However, 
if $a > 0$, then in order for this inequality to be satisfied, $a > \frac{1}{n}$ to get 
$0 < \frac{1}{n} < a < \epsilon$. This is a contradiction to the fact that $a \leq 1/n$ for all $n$.

\question For each of the following statments, determine whether it is true or false and
justify your answer

a. If the sequence $\{{a_n}^2\}$ converges, then the sequence $\{a_n\}$ converges.

If the sequence $\{{a_n}^2\}$ converges, then it is bounded. Since it is bounded, 
$\exists x \in \mathbb{R}$ such that $|a_n|^2 \leq x$. Thus, $|a_n| \leq \sqrt{x}$, therefore 
$\{a_n\}$ is bounded. Since $\{a_n\}$ is bounded, it converges.

b. If the sequence $\{a_n + b_n\}$ converges, then the sequences $\{a_n\}$ and $\{b_n\}$ also converge.

Let all elements of $\{a_n\}$ be real numbers $\mathbb{R}$ and all elements of $\{b_n\}$ be negative 
real numbers $\mathbb{R}^-$, then $\{a_n + b_n\}$ is will cancel out to 0 and thus converge. However, 
both sets are unbounded, so $\{a_n\}$ and $\{b_n\}$ do not converge

c. If the sequence $\{a_b + b_n\}$ and $\{a_n\}$ converge, then the sequence $\{b_n\}$ also converges.

Let $\{a_b + b_n\}$ converge to some value $x$ and $\{a_n\}$ converge to some value $y$. Then,

$$\lim_{n \to \infty} ((a_n + b_n) - (b_n)) = \lim_{n \to \infty} a_n$$

So,

\begin{align*}
    \lim_{n \to \infty} ((a_n + b_n) - (b_n)) &= \lim_{n \to \infty} (a_n + b_n) - \lim_{n \to \infty} (b_n) \\ 
    x &= y - \lim_{n \to \infty} (b_n) \\
    x - y &= -\lim_{n \to \infty} (b_n) \\
    \lim_{n \to \infty} (b_n) &= y - x
\end{align*}

So, $\{b_n\}$ converges to $y - x$.

d. If the sequence $\{|a_n|\}$ converges, then the sequence $\{a_n\}$ also converges.

Let $\{a_n\} = \{(-1)^n\}$. Then, $\{a_n\}$ does not converge but $\{|a_n|\}$ does converge ($\{|a_n|\} = {1}$ for all $n$). Thus,
this is false.  

\end{questions}

\end{document}
