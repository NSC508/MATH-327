\documentclass[addpoints]{exam}
\usepackage[legalpaper]{geometry}
\usepackage{etoolbox}
\usepackage[utf8]{inputenc}
\usepackage{amsmath, amsfonts}
\usepackage{enumerate}
\usepackage{hyperref}
\usepackage{relsize}
\newcommand{\sech}{\operatorname{sech}}
\newcommand{\R}{\mathbb{R}}
\newcommand{\Z}{\mathbb{Z}}
\newcommand{\lt}{\left}
\newcommand{\rt}{\right}
\newcommand{\pd}[2]{\frac{\partial #1}{\partial #2}}
\newcommand{\pdd}[3][]{\frac{\partial^{#1} #2}{\partial #3^{#1}}}
\newcommand{\lam}{\lambda}
\newcommand{\ansSpace}{\vspace{70mm}}


\title{Math 327 Homework 1}
\author{Sathvik Chinta}
\date{October 7th, 2022}

\begin{document}

\maketitle

\begin{questions}
\question Let $x$ and $y$ be two positive numbers. 

(i) Use the mathematical induction to show that if $x < y$, then $x^n < y^n$ for all
n $\in$ $\mathbb{N}$. 

First, we let $k = 1$. Given that $x < y$, $x^{k} < y^{k} = x^{1} < y^{1} = x < y$ which we are given so it is true. 

Now, we assume this is true for $k$. We want to show that it is true for $k+1$.

\[x^{k} < y^{k}\]

Since $x < y$, we can multiply both sides by $x$ to get

\[x^{k+1} < y^{k}x\]

Knowing that $x < y$, we can substitute $x$ for $y$ since the inequality will still hold. Thus, we can write

\[x^{k+1} < y^{k}y\]
\[x^{k+1} < y^{k+1}\]

(ii) Deduce that if $x^n < y^n$ for some $n \in \mathbb{N}$, then $x < y$.

Assume that $x^{n} < y^{n}$ for all $n$, but $x \geq y$. We then have two cases, 

Case 1: $x = y$.

If $x = y$. We can thus multiply both sides by $x$ and $y$ respectively (they are both equal, so 
the order is irrelevant) $n$ times to get $x^{n} = y^{n}$ for all $n$. This is a contradiction to our original statment of 
$x^{n} < y^{n}$, so $x \neq y$.

Case 2: $x > y$.

If $x > y$, we can multiply both sides by $y$ $n$ times to get 

\[xy^{n} > yy^{n}\]

Since $x^{n} < y^{n}$, we can substitute $y^{n}$ for $x^{n}$ since the inequality will still hold. Thus, we can write

\[xx^{n} > yy^{n}\]

\[x^{n+1} > y^{n+1}\]

For all $n$. However, if we plug in $n = n - 1$, we get 

\[x^{n} < y^{n}\]

which is a contradiction to our original statement of $x^{n} < y^{n}$, so $x$ cannot be less than $y$. 

Thus, we have shown that $x < y$.

\question Do problem 17 on page 11 of the textbook [F]. Define 

\[S = \{x \text{  }| \text{ } x \text{ in } \mathbb{R}, x \geq 0, x^2 < c \}\]

a. Show that $c + 1$ is an upper bound for $S$ and therefore, by the Completeness Axiom, $S$ has a least upper bound
that we denote by $b$.

We know that every element of the set $S$ must be greater than or equal to $0$. Furthermore, we know that every 
element in $x^2$ must be less than $c$. First, we prove that for all $x \geq 0$, $x \leq x^2$. When $x = 0$, we have 

\[0 \leq 0^2 = 0\]

When $x > 0$ (equivalent to $x \geq 1$), we can substitute $x^2$ for $xx$ to get

\[x \leq xx\]

Dividing both sides by $x$ gives us

\[1 \leq x\]

Which is the exact solution set we wanted. Thus, we have shown that for all $x \geq 0$, $x \leq x^2$.

Now, we can say that $x \leq x^2 < c$. For all integers $c$, $c + 1 > c$. Thus, we can write 

\[x \leq x^2 < c \leq c + 1\]

Thus, we know that $x < c + 1$ for all $x$ in $S$, so $c + 1$ is an upper bound for $S$.

By the completeness axiom, if $S$ has some upper bound, then there must be some least upper bound for the set 
$S$. Thus, there must exist some $b$ such that $b$ is an upper bound for $S$ and $b$ is the least upper bound for $S$.

b. Show that if $b^2 > c$, then we can choose a suitably small positive number $r$ such that $b - r$ is also an upper
bound for $S$, thus contradicting the choice of $b$ as the $least$ upper bound for $S$.

We know that $x^2 < c$ for all $x$ in $S$. If we take the square root from both sides, we find that the theoretical 
solution set is $x \in \mathbb{R}$ such that $x \geq 0$ and $x < \sqrt{c}$. Knowing that $b^2 > c$, we can take the 
square root from both sides of the inequality to get $b > \sqrt{c}$. Thus, we know that $b > \sqrt{c}$. We can then write
$x < \sqrt{c} < b$. If we take the difference between both sides, we get $x < b - \sqrt{c}$. Thus, we know that $b - \sqrt{c}$ 
must also be an upper bound for $S$. Since $c$ is positive, we know that $b - \sqrt{c} < b$. Thus, $b$ cannot be the $least$ upper
bound of $S$. 

c. Show that if $b^2 < c$, then we can choose a suitably small positive number $r$ such that $b + r$ belongs 
to $S$, thus contradicting the choice of $b$ as $an$ upper bound for $S$.

We assume that $b^2 < c$ and show that there must exist some $r$ such that $b + r$ belongs to $S$. We can expand $(b + r)^2$ as 

\[(b + r)^2 = b^2 + 2br + r\]

Let $r \leq b$, meaning that $r^2 \leq br$. Thus, we can write

\[(b + r)^2 = b^2 + 2br + r^2 \leq b^2 + 3br\]

Now, we asssumed $r \leq b$. We can further impose write $r < \frac{c - b^2}{3b}$ since $b^2 < c$ and $b < \sqrt{c}$. 

Thus, we get 

\[(b + r)^2 = b^2 + 2br + r^2 \leq b^2 + 3br < b^2 + 3b(\frac{c - b^2}{3b}) = b^2 + c - b^2 = c\]

Thus we have shown that $b + r$ belongs to $S$, meaning that $b + r < b$ which is a contradiction! So $b$ cannot be an 
upper bound for $S$.

d. Use parts (b) and (c) and the Positivity Axioms for $\mathbb{R}$ to conclude that $b^2 = c$. 

We know by the Completeness Axiom that $S$ must contain a least upper bound $b$. From part
(b), we know that $b^2 > c$ is a contradiction. Thus, $b^2 \leq 0$. From part (c), 
we know that $b^2 < c$ is a contradiction. Thus, $b^2 \geq c$. 

If $b^2 \leq c$, then $c - b^2$ is either positive or 0 by the Positivity Axioms. 

If $b^2 \geq c$, then $-(c - b^2)$ is either positive or 0 by the Positivity Axioms.

The only possibility, therefore, is that $c - b^2 = 0$ or $b^2 = c$. 

\question Suppose that $S$ is a non-empty set of real numbers that is bounded. Prove
that inf $S \leq$ sup $S$, and the quality holds if and only if $S$ consists of exactly
one number.

We must prove this both ways. We start with if inf $S \leq$ sup $S$, then $S$ consists of exactly one number.

We will prove by contradiciton. Assume that $S$ has more than one number in it. Then, there exists $x$ and $y$ 
in $S$ such that $x \neq y$. Thus, either $x < y$ or $x > y$. Assume that $x < y$. Since $S$ is bounded, there exists a lower bound 
inf $S$ such that every element of $S$ is greater than or equal to it. Thus, we can write inf $S \leq x < y$. 
Similarly, there exists an upper bound sup $S$ such that every element of $S$ is less than or equal to it. 
Thus, we can write inf $S \leq x < y \leq$ sup $S$. However, since the inequality between $x$ and $y$ is strict, 
inf $S <$ sup $S$ is also strict. This is a contradiction to our original statement of inf $S \leq$ sup $S$ since the
two values can never be equivalent, so $S$ must consist of exactly one number.

Now, we prove that if $S$ consists of exactly one number, then inf $S \leq$ sup $S$.

Since $S$ consists of exactly one number, we can write $S = \{x\}$. By definition, inf $S = x$ and sup $S = x$.
Thus, we can write inf $S = x =$ sup $S$. This is contained in inf $S \leq$ sup $S$ since inf $S \leq$ sup $S$ 
is an equivalence relation. Thus, we have shown that if $S$ consists of exactly one number, then inf $S \leq$ sup $S$.

Thus inf $S \leq$ sup $S$, and the quality holds if and only if $S$ consists of exactly one number.


\question Do Problem 10 on page 11 of the textbook [F].



\end{questions}

\end{document}
