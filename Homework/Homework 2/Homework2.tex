\documentclass[addpoints]{exam}
\usepackage[legalpaper]{geometry}
\usepackage{etoolbox}
\usepackage[utf8]{inputenc}
\usepackage{amsmath, amsfonts}
\usepackage{enumerate}
\usepackage{hyperref}
\usepackage{relsize}
\newcommand{\sech}{\operatorname{sech}}
\newcommand{\R}{\mathbb{R}}
\newcommand{\Z}{\mathbb{Z}}
\newcommand{\lt}{\left}
\newcommand{\rt}{\right}
\newcommand{\pd}[2]{\frac{\partial #1}{\partial #2}}
\newcommand{\pdd}[3][]{\frac{\partial^{#1} #2}{\partial #3^{#1}}}
\newcommand{\lam}{\lambda}
\newcommand{\ansSpace}{\vspace{70mm}}


\title{Math 327 Homework 1}
\author{Sathvik Chinta}
\date{October 7th, 2022}

\begin{document}

\maketitle

\begin{questions}
\question Let $x$ and $y$ be two positive numbers. 

(i) Use the mathematical induction to show that if $x < y$, then $x^n < y^n$ for all
n $\in$ $\mathbb{N}$. 

First, we let $k = 1$. Given that $x < y$, $x^{k} < y^{k} = x^{1} < y^{1} = x < y$ which we are given so it is true. 

Now, we assume this is true for $k$. We want to show that it is true for $k+1$.

\[x^{k} < y^{k}\]

Since $x < y$, we can multiply both sides by $x$ to get

\[x^{k+1} < y^{k}x\]

Knowing that $x < x^{k} < y^{k}$, we can substitute $x$ for $y^{k}$ since the inequality will still hold. Thus, we can write

\[x^{k+1} < y^{k}y^{k}\]
\[x^{k+1} < y^{k+1}\]

(ii) Deduce that if $x^n < y^n$ for some $n \in \mathbb{N}$, then $x < y$.

Assume that $x^{n} < y^{n}$ for all $n$, but $x \geq y$. We then have two cases, 

Case 1: $x = y$.

If $x = y$. We can thus multiply both sides by $x$ and $y$ respectively (they are both equal, so 
the order is irrelevant) $n$ times to get $x^{n} = y^{n}$ for all $n$. This is a contradiction to our original statment of 
$x^{n} < y^{n}$, so $x \neq y$.

Case 2: $x > y$.

If $x > y$, we can multiply both sides by $y$ $n$ times to get 

\[xy^{n} > yy^{n}\]

Since $x^{n} < y^{n}$, we can substitute $y^{n}$ for $x^{n}$ since the inequality will still hold. Thus, we can write

\[xx^{n} > yy^{n}\]

\[x^{n+1} > y^{n+1}\]

For all $n$. However, if we plug in $n = n - 1$, we get 

\[x^{n} < y^{n}\]

which is a contradiction to our original statement of $x^{n} < y^{n}$, so $x$ cannot be less than $y$. 

Thus, we have shown that $x < y$.

\question Do problem 17 on page 11 of the textbook [F]

\question Suppose that $S$ is a non-empty set of real numbers that is bounded. Prove
that inf $S \leq$ sup $S$, and the quality holds if and only if $S$ consists of exactly
one number.

\question Do Problem 10 on page 11 of the textbook [F].

\end{questions}

\end{document}
