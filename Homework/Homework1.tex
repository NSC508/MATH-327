\documentclass[addpoints]{exam}
\usepackage[legalpaper]{geometry}
\usepackage{etoolbox}
\usepackage[utf8]{inputenc}
\usepackage{amsmath, amsfonts}
\usepackage{enumerate}
\usepackage{hyperref}
\usepackage{relsize}
\newcommand{\sech}{\operatorname{sech}}
\newcommand{\R}{\mathbb{R}}
\newcommand{\Z}{\mathbb{Z}}
\newcommand{\lt}{\left}
\newcommand{\rt}{\right}
\newcommand{\pd}[2]{\frac{\partial #1}{\partial #2}}
\newcommand{\pdd}[3][]{\frac{\partial^{#1} #2}{\partial #3^{#1}}}
\newcommand{\lam}{\lambda}
\newcommand{\ansSpace}{\vspace{70mm}}


\title{Math 327 Homework 1}
\author{Sathvik Chinta}
\date{October 7th, 2022}

\begin{document}

\maketitle

\begin{questions}
\question Recall that in the textbook, a function f : A $\rightarrow$ B is said to be
invertible f is one-to-one and onto. (This is also called bijection in some literature.)
Show that a function f : A $\rightarrow$ B is invertible if and only if there is a function g :
B $\rightarrow$ A such that g $\circ$ f = identity function on A and f $\circ$ g = identity 
function on B.

A function $F$ is called the identity function given that $F(x) = x$. We are given that $f(g(x)) 
= x$ and $g(f(x)) = x$. We want to prove that $f$ is both one-to-one and onto. 

Let $f(g(x)) = f(g(y))$. Given that this is the identity function, we know that $f(g(x)) = x$ 
and $f(g(y)) = y$. As such, $x = y$. So, $f(g(x))$ is one-to-one. 

We know that $f(g(x)) = x$. Let $y \in B$. We want to show that for some $x \in A$, $f(g(x)) = y$.
Since $f(g(x))$ is the identity function, we know that $f(g(y)) = y$. As such, when $x = y$, 
$f(g(x)) = y$. So, $f(g(x))$ is onto.

With similar lines of reasoning as above, we know that $g(f(x))$ is both one-to-one and onto as well. 



\question Let f : A $\rightarrow$ B and g : B $\rightarrow$ C be functions.\\
(a) Prove that f and g are injective, then so is g $\circ$ f. 

Let $g(f(x)) = g(f(y))$. Since $g$ is injective, we know that $f(x) = f(y)$. Since $f$ is injective,
we know that $x = y$. So, $g(f(x))$ is injective.

(b) Suppose g $\circ$ f is surjective. Does it follow that f is surjective?

Let $a \in B$. Since $g \circ f$ is surjective, there exists $x \in A$ such that $g(f(x)) = a$. However, 
we do not know if $f(x) = a$. As such, $f$ is not surjective. 

(c) Suppose g $\circ$ f is surjective. Does it follow that g is surjective?

Let $a \in C$. Since g $\circ$ f is surjective, there exists $b \in B$ such that $g(f(b)) = a$. So, 
if $x = f(b) \in B$, then $g(x) = a$. As such, $g$ is surjective.

\question Let f: $A \rightarrow$ B. For a set $E \subset A$, the set $F(E) \subset B$ is defined by $F(E) = \{f(a): a \in E\}$, and 
is called the image of $E$. For $F \subset B$, the set $f^{-1}(F) \subset A$ is defined by $f^{-1}(F) = \{a \in A: f(a) \in F\}$ and is 
called the inverse image of $F$. \\
(i) Let $E \subset A$. Prove that $E \subset f^{-1}(f(E))$. Give an example to show that the inclusion may be strict. 
What happens when $f$ is injective? 

(ii) Let $\Lambda$ be a set and that each $\lambda \in \Lambda$, let $E_{\lambda}$ be a subset of $A$. Prove that 
\[f(\cup_{\lambda \in \Lambda}E_{\lambda}) = \cup_{\lambda \in \Lambda} f(E_\lambda) \text{ and } f(\cap_{\lambda \in \Lambda}E_{\lambda})
\subset \cup_{\lambda \in \Lambda}f(E_\lambda)\]

Give an example in which the last inclusion is proper. What happens when $f$ is injective?

\end{questions}

\end{document}
