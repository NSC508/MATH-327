\documentclass[addpoints]{exam}
\usepackage[legalpaper]{geometry}
\usepackage{etoolbox}
\usepackage[utf8]{inputenc}
\usepackage{amsmath, amsfonts}
\usepackage{enumerate}
\usepackage{hyperref}
\usepackage{relsize}
\newcommand{\sech}{\operatorname{sech}}
\newcommand{\R}{\mathbb{R}}
\newcommand{\Z}{\mathbb{Z}}
\newcommand{\lt}{\left}
\newcommand{\rt}{\right}
\newcommand{\pd}[2]{\frac{\partial #1}{\partial #2}}
\newcommand{\pdd}[3][]{\frac{\partial^{#1} #2}{\partial #3^{#1}}}
\newcommand{\lam}{\lambda}
\newcommand{\ansSpace}{\vspace{70mm}}


\title{Math 327 Homework 4}
\author{Sathvik Chinta}
\date{October 28th, 2022}

\begin{document}

\maketitle

\begin{questions}
\question Using only the Archimedean property of $\mathbb{R}$, give a direct $\epsilon - N$ verifictation of the convergence
of the following sequences

a. $\frac{2}{\sqrt{n}} + \frac{1}{n} + 3$

For any $\epsilon > 0$, by the Archimedean property there exists some $N \in \mathbb{N}$ with
$N > \frac{\sqrt{3}}{\epsilon}$. Then, for any $n \geq N$

\[n \leq \frac{3}{n^2} \leq \frac{3}{N^2} < \epsilon^2\]

So $|a_n - 3| < \epsilon$ where $a_n$ is the sequence. 

So, the sequence converges to 3. 

b. $\frac{n^2}{n^2 + n}$

Given $\epsilon > 0$, by the Archimedean property there exists some $N \in \mathbb{N}$ with
$N < \epsilon$. Then, for any $n \geq N$

\[n \geq \frac{1}{n} \geq \frac{1}{N} > \frac{1}{\epsilon}\]

So $|a_n - 1| < \epsilon$ where $a_n$ is the sequence.

So, the sequence converges to 1.

\question Prove that 

\[\lim_{x \to -\infty} n^{\frac{1}{n}} = 1\]

First, we define 

\[\alpha_n = n^{\frac{1}{n}} - 1\]

This implies 

$n^{\frac{1}{n}} = 1 + \alpha_n $

Raising both sides to the $n$th power, we get

$n = (1 + \alpha_n)^n$

Doing binomaial expansion to the right hand side, we get

$n = 1 + n\alpha_n + \frac{n(n-1)}{2}\alpha_n^2 + \frac{n(n-1)(n-2)}{6}\alpha_n^3 + \cdots$

Notice that $n \geq \frac{n(n-1)}{2}\alpha_n^2$. Dividing both sides by $n$, we get 

$1 \leq \frac{n-1}{2}\alpha_n^2$. Re-arranging the terms, we have 

$\alpha_n^2 \leq \frac{2}{n-1}$. 

For any $\epsilon > 0$, by the Archimedean property there exists some $N \in \mathbb{N}$ with
$N > \frac{2}{\epsilon^2} + 1$. Then, for any $n \geq N$

\[\alpha_n^2 \leq \frac{2}{n - 1} \leq \frac{2}{N - 1} < \epsilon^2\]

So, $|\alpha_n - 0| < \epsilon$ and thus the limit of $\alpha_n$ as $n \to \infty$ is $0$.

Since we defined $\alpha_n$ as $n^{\frac{1}{n}} - 1$, we know that the limit of
$n^{\frac{1}{n}}$ as $n \to \infty$ is $1$.

\question Suppose that the sequence $\{a_n\}$ converges to $a$ and that $|a| < 1$. 
Prove that the sequence $\{(a_n)^n\}$ converges to 0. 

Given that $|a| < 1$, we know that there must exist some $\epsilon \in \mathbb{R}$ such that
$\epsilon = \frac{1 - |a|}{2}$. 

Thus, there exists some $N \in \mathbb{N}$ so that $|a_n - a| < \epsilon$ for all $n \geq N$. This
implies that $|a_n| < |a + \epsilon|$ for all $n \geq N$. This, in turn, implies 
$|a_n|^n < |a + \epsilon|^n$.

By proposition 2.28 in the textbook, since $|a + \epsilon| < 1$, $|a + \epsilon|^n$ converges to 0. Since
$|a_n|^n < |a + \epsilon|^n$ for all $n \geq N$, we know that $(a_n)^n$ converges to 0 as well. 

\question

We define $D$ as a set of sequences. If we write 

$n_0n_1n_2...$ for all $n_k \in D$, then we get $n_0.012345678012345678\dots$. Since 
every element of D is encompassed in this decimal, it gives an order on D. If we define $j$ as 
the smallest non-negative integer such that $n_j \neq m_j$, we can multiple both $n_0.n_1n_2n_3\dots$ and
$m_0.m_1m_2m_3\dots$ by $10^j$ to get $n_0n_1n_2\dots n_j.n_{j+1}\dots$ and $m_0m_1m_2\dots m_j.m_{j+1}\dots$.
Since $n_j \neq m_j$. Using this definition, the $\prec$ operator behaves the same as the inequality 
operator, thus making D a linear order and making it so that if $x, y \in D$, then only one of the 
following statements is true: $x \prec y$, $x \succ y$, or $x = y$. Furthermore, if $x, y, z \in D$ and 
$x \prec y$ and $y \prec z$, then $x \prec z$.

\question Properties


i. If $n_k > 9$, then $\frac{n_1}{10} \geq 1$, thus in the boundary $(x_0, x_1)$ there must exist some
integer $> x$, which is a contradiction to our initial conditions.  

ii. For any given $x_k$, $x_k < x$. Thus, $0 \leq x - x_k$. Since the interval boundary between the two can 
never contain an integer, we know that $x - x_k < 1$. As well. Since we state that $x_0 = n_0$,
$x_k$ can be expanded as $x_k = \frac{n_k}{10^k} + \frac{n_{k- 1}}{10^{k - 1}} + \frac{n_{k- 2}}{10^{k - 2}} + \dots + n_0 \leq \frac{1}{10^k}$.
Thus, we can write $0 \leq x - x_k \leq \frac{1}{10^k}$.

iii. Since $n_k \in \{0, 1, 2, \dots 9\}$ for all $k \in \mathbb{N}$, 
and $k \geq 0$, $\frac{n_k}{10^k} > 0$, so $x_{k-1} \leq x_k$ for all $k \in \mathbb{N}$.

iv. This sequence is non-repeating, so there must exist some $k > j$ such that $n_k \neq 9$ for all 
$j \in \mathbb{N}$. Furthermore, we have already established $n_k \in {0, 1, 2, \dots 9}$ and since 
$n_0$ is an integer it follows that $n_0 \in \mathbb{Z}$. Thus, the set is the same as the set of D
from the previous question.

v. Since the sequence is monotonically increasing, if x < y in $\mathbb{R}$, this implies the 
choice of the initial $n_0$ is smaller. Thus, $x < y$ implies $D(x) \prec D(y)$ and vice versa.

\question 

i. For each of the following statements, determine whether it is true or false and justify your answer

a. Every bounded sequence converges

False. Let the sequence consist of entirely 0s and 1s, each alternating in occurance. In this case, 
bounds are known but the sequence does not converge.

b. A convergent sequence of positive numbers has a positive limit

True, a negative limit would imply that the sequence is bounded below by a negative number and that 
the sequence is monotonically decreasing. If the set only containts positive numbers, 0 will always 
be a lower bound and thus the lowest possible value for a limit is 0. 

c. The sequence $\{n^2 + 1\}$ converges

No, the sequence will tend towards infinity as $n \to \infty$.

d. A convergent sequence of rational numbers has a rational limit

False, a convergent sequence of rational numbers can have an irrational limit. For example, the sequence
$(1 + \frac{1}{n})^n$ converges to $e$.

e. The limit of a convergent sequnce in the interval (a, b) also belongs to (a, b)

True, if a sequence is in an interval and it converges to a number in that interval, then the limit must
also be in that interval.

ii. Show that the set of irrational numbers fails to be closed

We let $x$ to be a rational number. We let $a_n$ be the sequence of irrational numbers such that 
$a_n$ converges to $x$. Since the set of irrational numbers cannot contain $x$, but there exists a 
sequence within the set of irrational numbers that converges to $x$, the set of irrational numbers
cannot be closed.

\end{questions}

\end{document}
