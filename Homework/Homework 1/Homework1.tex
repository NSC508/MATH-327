\documentclass[addpoints]{exam}
\usepackage[legalpaper]{geometry}
\usepackage{etoolbox}
\usepackage[utf8]{inputenc}
\usepackage{amsmath, amsfonts}
\usepackage{enumerate}
\usepackage{hyperref}
\usepackage{relsize}
\newcommand{\sech}{\operatorname{sech}}
\newcommand{\R}{\mathbb{R}}
\newcommand{\Z}{\mathbb{Z}}
\newcommand{\lt}{\left}
\newcommand{\rt}{\right}
\newcommand{\pd}[2]{\frac{\partial #1}{\partial #2}}
\newcommand{\pdd}[3][]{\frac{\partial^{#1} #2}{\partial #3^{#1}}}
\newcommand{\lam}{\lambda}
\newcommand{\ansSpace}{\vspace{70mm}}


\title{Math 327 Homework 1}
\author{Sathvik Chinta}
\date{October 7th, 2022}

\begin{document}

\maketitle

\begin{questions}
\question Recall that in the textbook, a function f : A $\rightarrow$ B is said to be
invertible f is one-to-one and onto. (This is also called bijection in some literature.)
Show that a function f : A $\rightarrow$ B is invertible if and only if there is a function g :
B $\rightarrow$ A such that g $\circ$ f = identity function on A and f $\circ$ g = identity 
function on B.

We first prove that if there exists a function B $\rightarrow$ A such that g $\circ$ 
f = identity function on A and f $\circ$ g = identity function on B, then $f$ is invertible.

A function $F$ is called the identity function given that $F(x) = x$. We are given that $f(g(x)) 
= x$ and $g(f(x)) = x$. We want to prove that $f$ is both one-to-one and onto. 

If $f(x) = f(y)$, then $g(f(x)) = g(f(y))$. Thus, $x = y$. Thus, $f$ is one-to-one.

For any $a \in A$, we have $a = f(g(a))$. Thus, $f(a) \in B$. Thus, $f$ is onto.

Since $f$ is one-to-one and onto, $f$ is invertible.

Now, we attempt to prove that if $f$ is invertible, then there exists a function $g$ such that 
$g \circ f$ = identity function on A and $f \circ g$ = identity function on B.

We shall prove by contradiction. Suppose that $f$ is invertible, but there does not exist a
function $g$ such that $g \circ f$ = identity function on A and $f \circ g$ = identity function
on B. However, since $f$ is invertible, $f$ is one-to-one and onto. As such, there exists a function 
mapping $f(x) = y$ for some $y \in B$. Thus, there exists a function $g$ such that $g(y) = x$. Thus, 
$g \circ f$ = identity function on A and $f \circ g$ = identity function on B. Thus, we have reached a
contradiction. Thus, $f$ is invertible if and only if there exists a function $g$ such that $g \circ f$


\question Let f : A $\rightarrow$ B and g : B $\rightarrow$ C be functions.\\
(a) Prove that f and g are injective, then so is g $\circ$ f. 

Let $g(f(x)) = g(f(y))$. Since $g$ is injective, we know that $f(x) = f(y)$. Since $f$ is injective,
we know that $x = y$. So, $g(f(x))$ is injective.

(b) Suppose g $\circ$ f is surjective. Does it follow that f is surjective?

Let $a \in B$. Since $g \circ f$ is surjective, there exists $x \in A$ such that $g(f(x)) = a$. However, 
we do not know if $f(x) = a$. As such, $f$ is not surjective. 

(c) Suppose g $\circ$ f is surjective. Does it follow that g is surjective?

Let $a \in C$. Since g $\circ$ f is surjective, there exists $b \in B$ such that $g(f(b)) = a$. So, 
if $x = f(b) \in B$, then $g(x) = a$. As such, $g$ is surjective.

\question Let f: $A \rightarrow$ B. For a set $E \subset A$, the set $f(E) \subset B$ is defined by $f(E) = \{f(a): a \in E\}$, and 
is called the image of $E$. For $F \subset B$, the set $f^{-1}(F) \subset A$ is defined by $f^{-1}(F) = \{a \in A: f(a) \in F\}$ and is 
called the inverse image of $F$. \\
(i) Let $E \subset A$. Prove that $E \subset f^{-1}(f(E))$. Give an example to show that the inclusion may be strict. 
What happens when $f$ is injective? 

Let $a \in E$. Since $E \subset f^{-1}(f(E))$. We know that $f(a) \in f(E)$ by the definition of $f(E)$. As such, 
$f(a) \in f^{-1}(f(E))$. So, $a \in f^{-1}(f(E))$. 

If there were two elements in E that mapped to the same $f(a)$, however, then our inclusion would be strict ($f^{-1}(f(E))$ would be greater in size).

If $f$ were injective, however, this would not be possible, meaning that $f^{-1}(f(E)) = E$.

(ii) Let $\Lambda$ be a set and that each $\lambda \in \Lambda$, let $E_{\lambda}$ be a subset of $A$. Prove that 
\[f(\cup_{\lambda \in \Lambda}E_{\lambda}) = \cup_{\lambda \in \Lambda} f(E_\lambda) \text{ and } f(\cap_{\lambda \in \Lambda}E_{\lambda})
\subset \cap_{\lambda \in \Lambda}f(E_\lambda)\]

Give an example in which the last inclusion is proper. What happens when $f$ is injective?

First, we prove $f(\cup_{\lambda \in \Lambda}E_{\lambda}) = \cup_{\lambda \in \Lambda} f(E_\lambda)$. 

Let $\Lambda = \{\lambda_1, \lambda_2, ..., \lambda_n\}$ where $n$ is the size of $\Lambda$. We can expand the inside of the left equation to be

\[f(\cup_{\lambda \in \Lambda}E_{\lambda}) = f(E_{\lambda_1} \cup E_{\lambda_2} \cup ... E_{\lambda_n})\]

Now, we know that each $E_{\lambda_i}$ is a subset of $A$. Since $E_{\lambda_i}$ itself is a set, we can denote its members as
$E_{\lambda_i} = \{{E_{\lambda_i}}_1, {E_{\lambda_i}}_2, ..., {E_{\lambda_i}}_m\}$ where $m$ is the size of $E_{\lambda_i}$. 

By the definition of $f(E)$, we can then denote $f(E_{\lambda_i})$ as 
$f(E_{\lambda_i}) = \{f({E_{\lambda_i}}_1), f({E_{\lambda_i}}_2), ..., f({E_{\lambda_i}}_m)\}$.

Using the definition of union, we can denote the inner part of the right hand side as

\[E_{\lambda_1} \cup E_{\lambda_2} \cup ... E_{\lambda_n} = \{{E_{\lambda_1}}_1, {E_{\lambda_1}}_2, ..., {E_{\lambda_1}}_{m1}, {E_{\lambda_2}}_1, {E_{\lambda_2}}_2, ..., {E_{\lambda_2}}_{m2}, ..., {E_{\lambda_n}}_{mn}\}\]

Thus, $f(\cup_{\lambda \in \Lambda}E_{\lambda}) = \{f({E_{\lambda_1}}_1), f({E_{\lambda_1}}_2), ..., f({E_{\lambda_1}}_{m1}), f({E_{\lambda_2}}_1), f({E_{\lambda_2}}_2), ..., f({E_{\lambda_2}}_{m2}), ..., f({E_{\lambda_n}}_{mn})\}$

Now, we can expand the right hand side to be

\[\cup_{\lambda \in \Lambda} f(E_\lambda) = f(E_{\lambda_1}) \cup f(E_{\lambda_2}) \cup f(E_{\lambda_n})\]
\[ = \{f({E_{\lambda_1}}_1), f({E_{\lambda_1}}_2), ..., f({E_{\lambda_1}}_{m1})\} \cup \{f({E_{\lambda_2}}_1), f({E_{\lambda_2}}_2), ..., f({E_{\lambda_2}}_{m2})\} \cup \{f({E_{\lambda_n}}_1), f({E_{\lambda_n}}_2), ..., f({E_{\lambda_n}}_{mn})\}\]
\[= \{f({E_{\lambda_1}}_1), f({E_{\lambda_1}}_2), ..., f({E_{\lambda_1}}_{m1}), f({E_{\lambda_2}}_1), f({E_{\lambda_2}}_2), ..., f({E_{\lambda_2}}_{m2}), ..., f({E_{\lambda_n}}_{mn})\}\]

Which we can see is the same as the left hand side. Thus, $f(\cup_{\lambda \in \Lambda}E_{\lambda}) = \cup_{\lambda \in \Lambda} f(E_\lambda)$.

Next, we prove $f(\cap_{\lambda \in \Lambda}E_{\lambda}) \subset \cap_{\lambda \in \Lambda}f(E_\lambda)$.

Let $a \in f(\cap_{\lambda \in \Lambda}E_{\lambda})$. We can expand the inside of the right hand side to be

\[f(\cap_{\lambda \in \Lambda}E_{\lambda}) = f(E_{\lambda_1} \cap E_{\lambda_2} \cap ... E_{\lambda_n})\]

In the trivial case, there are no shared elements between the $E_{\lambda_i}$, so $f(\cap_{\lambda \in \Lambda}E_{\lambda}) = \emptyset$.

Similarly, for the right hand side we can expand it to be

\[\cap_{\lambda \in \Lambda}f(E_\lambda) = f(E_{\lambda_1}) \cap f(E_{\lambda_2}) \cap f(E_{\lambda_n})\]

If we assume the trivial case again, then $f(\cap_{\lambda \in \Lambda}E_{\lambda}) = \emptyset$, so $f(\cap_{\lambda \in \Lambda}E_{\lambda}) \subset \cap_{\lambda \in \Lambda}f(E_\lambda)$.

If we assume the non-trivial case, then there exists some overlap between two or more of the $E_{\lambda_i}$ with $a$ being an element in said overlap ($a$ is present in two or more $f(E_{\lambda_i}$)). 

Expanding out the right hand side, we can see that $a$ is an element of $f(E_{\lambda_1}) \cap f(E_{\lambda_2}) \cap f(E_{\lambda_n})$ since $a$ is present in two or more $f(E_{\lambda_i})$. 

Thus $a \in \cap_{\lambda \in \Lambda}f(E_\lambda)$, so $f(\cap_{\lambda \in \Lambda}E_{\lambda}) \subset \cap_{\lambda \in \Lambda}f(E_\lambda)$.

Since $f$ is being applied before the intersection on the left hand side, but after the intersection on the right hand side, we know that the inclusion may be proper.

If, for instance, there were no common elements between the $E_{\lambda_i}$, then $f(\cap_{\lambda \in \Lambda}E_{\lambda}) = \emptyset$. However, assume that two elements (that are not equal) map
to the same element in $f(E_{\lambda_i})$. Then, $f(\cap_{\lambda \in \Lambda}E_{\lambda})$ would contain that subset. As such, the inclusion in this case is proper. 

If $f$ were injective, however, we would not have this problem. In this case, $f(\cap_{\lambda \in \Lambda}E_{\lambda}) = \cap_{\lambda \in \Lambda}f(E_\lambda)$.

\question Let $X$ and $A$ be two sets. Here $A$ serves as an indexing set so that for each $\lambda \in \Lambda$, $A_\lambda$ is a subset of $X$. 
Suppose $B \subset X$. Show that:

(a) $B \cap (\cup_{\lambda \in \Lambda}A_{\lambda}) = \cup_{\lambda \in \Lambda}(B \cap A_{\lambda})$

Lets expand the inner part of the left hand side of the equation. We have

\[\cup_{\lambda \in \Lambda}A_{\lambda} = A_{\lambda_1} \cup A_{\lambda_2} ... \cup A_{\lambda_n}\]

Where $n$ is the size of $\Lambda$. This simplifies to $A_\Lambda$. Taking the intersection of this set with B, we then get
$B \cap A_\Lambda$. This will only contain elements that are in both $B$ and $A_\Lambda$.

Looking at the right side of the equation, we then get the following:

\[\cup_{\lambda \in \Lambda}(B \cap A_{\lambda}) = (B \cap A_{\lambda_1}) \cup (B \cap A_{\lambda_2}) ... \cup (B \cap A_{\lambda_n})\]

Where $n$ is the size of $\Lambda$. Each individual term will only contain elements that are in both $B$ and $A_{\lambda_i}$. Thus, taking the
union of these terms will only contain elements that are in both $B$ and $A_\Lambda$. Thus, both are equal. 

(b) $B \cup (\cap_{\lambda \in \Lambda}A_{\lambda}) = \cap_{\lambda \in \Lambda}(B \cup A_{\lambda})$

Lets expand the inner part of the left hand side of the equation. We have

\[\cap_{\lambda \in \Lambda}A_{\lambda} = A_{\lambda_1} \cap A_{\lambda_2} ... \cap A_{\lambda_n}\]

Where $n$ is the size of $\Lambda$. This will contain elements that are in all of the $A_{\lambda_i}$. We will denote this set as $S$. 
Taking the union of this set with B, we then get $B \cup S$. This will contain elements that are in either $B$ or $S$ or both.

Looking at the right side of the equation, we then get the following:

\[\cap_{\lambda \in \Lambda}(B \cup A_{\lambda}) = (B \cup A_{\lambda_1}) \cap (B \cup A_{\lambda_2}) ... \cap (B \cup A_{\lambda_n})\]

Where $n$ is the size of $\Lambda$. Each individual term will contain elements that are in either $B$ or $A_{\lambda_i}$ or both. Thus, taking the
intersection of these terms will contain elements that are in either $B$ or $S$ or both. Thus, both are equal.

(c) $(\cup_{\lambda \in \Lambda}A_{\lambda})^c = \cap_{\lambda \in \Lambda}(A_{\lambda})^c$

Lets expand the inner part of the left hand side of the equation. We have

\[(\cup_{\lambda \in \Lambda}A_{\lambda})^c = (A_{\lambda_1} \cup A_{\lambda_2} ... \cup A_{\lambda_n})^c\]

Where $n$ is the size of $\Lambda$. This simplifies to $A_\Lambda$. Taking the complement of this set, we then get
$(A_\Lambda)^c$. This will only contain elements that are not in $A_\Lambda$.

Looking at the right side of the equation, we then get the following:

\[\cap_{\lambda \in \Lambda}(A_{\lambda})^c = (A_{\lambda_1})^c \cap (A_{\lambda_2})^c ... \cap (A_{\lambda_n})^c\]

Where $n$ is the size of $\Lambda$. Each individual term will only contain elements that are not in any $A_{\lambda_i}$. Thus, taking the
intersection of these terms will only contain elements that are not in $A_\Lambda$. Thus, both are equal.

(d) $(\cap_{\lambda \in \Lambda}A_{\lambda})^c = \cup_{\lambda \in \Lambda}(A_{\lambda})^c$

Lets expand the inner part of the left hand side of the equation. We have

\[(\cap_{\lambda \in \Lambda}A_{\lambda})^c = (A_{\lambda_1} \cap A_{\lambda_2} ... \cap A_{\lambda_n})^c\]

Where $n$ is the size of $\Lambda$. This will contain elements that are in all of the $A_{\lambda_i}$. We will denote this set as $S$.

Taking the complement of this set, we then get $(S)^c$. This will only contain elements that are not in $S$.

Looking at the right side of the equation, we then get the following:

\[\cup_{\lambda \in \Lambda}(A_{\lambda})^c = (A_{\lambda_1})^c \cup (A_{\lambda_2})^c ... \cup (A_{\lambda_n})^c\]

Where $n$ is the size of $\Lambda$. Each individual term will only contain elements that are not in $A_{\lambda_i}$. Thus, taking the

union of these terms will only contain elements that are not in $S$. Thus, both are equal.



\end{questions}

\end{document}
