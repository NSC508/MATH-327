\documentclass[addpoints]{exam}
\usepackage[legalpaper]{geometry}
\usepackage{etoolbox}
\usepackage[utf8]{inputenc}
\usepackage{amsmath, amsfonts}
\usepackage{enumerate}
\usepackage{hyperref}
\usepackage{relsize}
\newcommand{\sech}{\operatorname{sech}}
\newcommand{\R}{\mathbb{R}}
\newcommand{\Z}{\mathbb{Z}}
\newcommand{\lt}{\left}
\newcommand{\rt}{\right}
\newcommand{\pd}[2]{\frac{\partial #1}{\partial #2}}
\newcommand{\pdd}[3][]{\frac{\partial^{#1} #2}{\partial #3^{#1}}}
\newcommand{\lam}{\lambda}
\newcommand{\ansSpace}{\vspace{70mm}}


\title{Math 327 Homework 7}
\author{Sathvik Chinta}
\date{November 29th, 2022}

\begin{document}

\maketitle

\begin{questions}
\question \textbf{10} \\

We want to prove that if a function $f: (a, b) \in \mathbb{R}$
is uniformly continuous, then it is bounded. \\

By the $\epsilon - \delta$ definition of continuity, 
a function is continuous in the domain $D$ provided that for
every positive number $\epsilon$ there is a positive number
$\delta$ for which 

\[|f(x) - f(y)| < \epsilon \quad \text{if} \quad |x - y| < \delta\]

Let $f$ be a function on $(a, b)$ that is uniformly continuous.\\

We prove by contradiction. Let $f$ be a function on $(a, b)$ that is uniformly continuous but not bounded.\\

Then, for all $M > 0$, there exists $x \in (a, b)$ such that $|f(x)| > M$.\\

Then, $f(x) > M$ or $f(x) < -M$. Then, let $M = \epsilon + f(y)$ where $|x - y| < \delta$.

Then, $f(x) > \epsilon + f(y)$ or $f(x) < -\epsilon - f(y)$. This is equivalent to saying
that $f(x) - f(y) > \epsilon$ or $f(x) - f(y) < -\epsilon$. In other words:

\[|f(x) - f(y)| > \epsilon\]

However, this is a contradiction to the fact that $f$ is uniformly continuous.\\



\question \textbf{7} \\
We are given that $f: [0, 1] \rightarrow \mathbb{R}$ is continuous,
$f(0) > 0$ and $f(1) = 0$. We want to prove that there is a number
$x_0 \in (0, 1]$ such that $f(x_0) = 0$ and $f(x) > 0$ for 
$0 \leq x < x_0$. 

We prove by contradiction. Let $f: [0, 1] \rightarrow \mathbb{R}$ be continuous,
$f(0) > 0$, $f(1) = 0$, and $x_0 \in (0, 1]$. We say that there is no
value $x_0$ for which $f(x_0) = 0$ and $f(x) > 0$ for 
$0 \leq x < x_0$. Let $x_0 = 1$ and $x = 0$. Then, $0 \leq x < x_0$ with
$x_0 \in (0, 1]$. However, $f(x_0) = 0$ and $f(x) > 0$ from our
assumptions. This is a contradiction. Thus, there is a value $x_0 \in (0, 1]$
such that $f(x_0) = 0$ and $f(x) > 0$ for $0 \leq x < x_0$.

\question \textbf{4} \\
We are given that $f: [-1, 1] \rightarrow \mathbb{R}$ is 
continuous, $f(-1) > -1$ and $f(1) < 1$. We want to prove that
$f$ has a fixed point where a fixed point is the point at which the
line $y = x$ intersects the graph of $f$.

Let's define another function $g: [-1, 1] \rightarrow \mathbb{R}$ by
$g(x) = f(x) - x$. Then, $g$ is continuous since $f$ is continuous and
$x$ is continous. We also have that $g(-1) = f(-1) - (-1) > 0$ and
$g(1) = f(1) - 1 < 0$. 

In other words

\[g(1) < 0 < g(-1)\]

By the Intermediate Value Theorem, there is a number $x_0 \in (-1, 1)$
such that $g(x_0) = 0$. Then, $f(x_0) = x_0$ since $g(x) = f(x) - x$.

Thus, $f$ has a fixed point.

\question \\
\textbf{1} \\
\textbf{a}\\
False. Let a function be defined as $f(x) = \frac{1}{x - 0.5}$. Then, this function does not have a maximum on the interval [0, 1]\\ 
\textbf{b}\\
Yes, if a functionis continous on a closed interval, it must have a minimum. If it were to not have a minimum, then it cannot be continous since
it would approach infinty\\
\textbf{c}\\
No, if the function has it's maximum at x = 1, then the the interval (0, 1) does not contain the maximum. \\
\textbf{d}\\
No, the function itself is unbounded so its image cannot be bounded. \\
\textbf{e}\\
Yes, if the image is bounded below there must be some $x \in (0, 1)$ such that $f(x) = \text{minimum}$. \\

\textbf{2} \\
\textbf{a}\\
x = 1 is the maximizer for this function. \\
\textbf{b}\\
x = 0 is the maximizer for this function \\
\textbf{c}\\
x = -1 is the maximizer for this function\\

\question \textbf{1} \\
\textbf{a} \\
No, let $f(x) = 1$ be our function. Then, $f(\mathbb{R}) = 1$.\\ 
\textbf{b} \\
No, if the function is unbounded then its image cannot be an interval\\
\textbf{c} \\
No, if the function is unbounded then its image cannot be an interval\\
\textbf{d} \\
Yes, since $f(0) < f(x) < f(1)$ for all $0 < x < 1$, the image is the interval [f(0), f(1)]\\

\question \textbf{3} \\

Let $f(x) = \frac{1}{\sqrt{x + x^2}} + x^2 - 2x$. Notice that $f(1) < 0$ and $f(2) > 0$. Thus, there must exist some $x \in (1, 2)$ for
which $f(x) = 0$ by the Intermediate Value Theorem. 
\end{questions}

\end{document}
