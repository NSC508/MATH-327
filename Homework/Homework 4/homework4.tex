\documentclass[addpoints]{exam}
\usepackage[legalpaper]{geometry}
\usepackage{etoolbox}
\usepackage[utf8]{inputenc}
\usepackage{amsmath, amsfonts}
\usepackage{enumerate}
\usepackage{hyperref}
\usepackage{relsize}
\newcommand{\sech}{\operatorname{sech}}
\newcommand{\R}{\mathbb{R}}
\newcommand{\Z}{\mathbb{Z}}
\newcommand{\lt}{\left}
\newcommand{\rt}{\right}
\newcommand{\pd}[2]{\frac{\partial #1}{\partial #2}}
\newcommand{\pdd}[3][]{\frac{\partial^{#1} #2}{\partial #3^{#1}}}
\newcommand{\lam}{\lambda}
\newcommand{\ansSpace}{\vspace{70mm}}


\title{Math 327 Homework 4}
\author{Sathvik Chinta}
\date{October 28th, 2022}

\begin{document}

\maketitle

\begin{questions}
\question Using only the Archimedean property of $\mathbb{R}$, give a direct $\epsilon - N$ verifictation of the convergence
of the following sequences

a. $\frac{2}{\sqrt{n}} + \frac{1}{n} + 3$

For any $\epsilon > 0$, by the Archimedean property there exists some $N \in \mathbb{N}$ with
$N > \frac{\sqrt{3}}{\epsilon}$. Then, for any $n \geq N$

\[n \leq \frac{3}{n^2} \leq \frac{3}{N^2} < \epsilon^2\]

So $|a_n - 3| < \epsilon$ where $a_n$ is the sequence. 

So, the sequence converges to 3. 

b. $\frac{n^2}{n^2 + n}$

Given $\epsilon > 0$, by the Archimedean property there exists some $N \in \mathbb{N}$ with
$N < \epsilon$. Then, for any $n \geq N$

\[n \geq \frac{1}{n} \geq \frac{1}{N} > \frac{1}{\epsilon}\]

So $|a_n - 1| < \epsilon$ where $a_n$ is the sequence.

So, the sequence converges to 1.

\question Prove that 

\[\lim_{x \to -\infty} n^{\frac{1}{n}} = 1\]

First, we define 

\[\alpha_n = n^{\frac{1}{n}} - 1\]

This implies 

$n^{\frac{1}{n}} = 1 + \alpha_n $

Raising both sides to the $n$th power, we get

$n = (1 + \alpha_n)^n$

Doing binomaial expansion to the right hand side, we get

$n = 1 + n\alpha_n + \frac{n(n-1)}{2}\alpha_n^2 + \frac{n(n-1)(n-2)}{6}\alpha_n^3 + \cdots$

Notice that $n \geq \frac{n(n-1)}{2}\alpha_n^2$. Dividing both sides by $n$, we get 

$1 \leq \frac{n-1}{2}\alpha_n^2$. Re-arranging the terms, we have 

$\alpha_n^2 \leq \frac{2}{n-1}$. 

For any $\epsilon > 0$, by the Archimedean property there exists some $N \in \mathbb{N}$ with
$N > \frac{2}{\epsilon^2} + 1$. Then, for any $n \geq N$

\[\alpha_n^2 \leq \frac{2}{n - 1} \leq \frac{2}{N - 1} < \epsilon^2\]

So, $|\alpha_n - 0| < \epsilon$ and thus the limit of $\alpha_n$ as $n \to \infty$ is $0$.

Since we defined $\alpha_n$ as $n^{\frac{1}{n}} - 1$, we know that the limit of
$n^{\frac{1}{n}}$ as $n \to \infty$ is $1$.

\question Suppose that the sequence $\{a_n\}$ converges to $a$ and that $|a| < 1$. 
Prove that the sequence $\{(a_n)^n\}$ converges to 0. 

Given that $|a| < 1$, we know that there must exist some $\epsilon \in \mathbb{R}$ such that
$\epsilon = \frac{1 - |a|}{2}$. 

Thus, there exists some $N \in \mathbb{N}$ so that $|a_n - a| < \epsilon$ for all $n \geq N$. This
implies that $|a_n| < |a + \epsilon|$ for all $n \geq N$. This, in turn, implies 
$|a_n|^n < |a + \epsilon|^n$.

By proposition 2.28 in the textbook, since $|a + \epsilon| < 1$, $|a + \epsilon|^n$ converges to 0. Since
$|a_n|^n < |a + \epsilon|^n$ for all $n \geq N$, we know that $(a_n)^n$ converges to 0 as well. 

\end{questions}

\end{document}
