\documentclass[addpoints]{exam}
\usepackage[legalpaper]{geometry}
\usepackage{etoolbox}
\usepackage[utf8]{inputenc}
\usepackage{amsmath, amsfonts}
\usepackage{enumerate}
\usepackage{hyperref}
\usepackage{relsize}
\newcommand{\sech}{\operatorname{sech}}
\newcommand{\R}{\mathbb{R}}
\newcommand{\Z}{\mathbb{Z}}
\newcommand{\lt}{\left}
\newcommand{\rt}{\right}
\newcommand{\pd}[2]{\frac{\partial #1}{\partial #2}}
\newcommand{\pdd}[3][]{\frac{\partial^{#1} #2}{\partial #3^{#1}}}
\newcommand{\lam}{\lambda}
\newcommand{\ansSpace}{\vspace{70mm}}


\title{Math 327 Homework 6}
\author{Sathvik Chinta}
\date{November 18th, 2022}

\begin{document}

\maketitle

\begin{questions}
\question \textbf{7} \textbf{a}\\

We want to prove that the function is continuous on the interval $[0,1]$.\\

Let ${x_n}$ be a sequence in $(0,1]$ such that $x_n \to x_0 \in [0, 1]$. By the sum and product properties of convergent
sequences, we have

\begin{align*}
\lim_{n \to \infty} f(x_n) &= \lim_{n \to \infty} \sqrt{x_n}\\
&= \sqrt{x_0} \\ 
&= f(x_0)
\end{align*}

When $x_0 = 0$, 

\begin{align*}
    \lim_{n \to \infty} f(x_n) &= \lim_{n \to \infty} \sqrt{x_0}\\
    &= 0 \\
    &= f(x_0)
\end{align*}

Thus, $f$ is continuous at $x_0$, thus $f$ is continous on the interval $[0, 1]$\\



\textbf{b}\\

We want to prove that the function is uniformly continuous on the interval $[0,1]$.\\

Let ${u_n}$ and ${v_n}$ be sequences in $[0, 1]$ such that

\[lim_{n \to \infty} [u_n - v_n] = 0\]

We want to prove that 

\[lim_{n \to \infty} |f(u_n) - f(v_n)| = 0\]

We shall prove so by arguing the contradition. Suppose that the differences between the two limits is not equal
to 0. Then, thre must exist some $\epsilon > 0$ such that

\[|f(u_n) - f(v_n)| \geq \epsilon\]

for all $n$.\\

We know, however, that the domain of $f$ is $[0, 1]$. By the Sequential Compactness Theorem,
there exists a subsequence ${u_{n_k}}$ of ${u_{n}}$ and a point $x_0$ in $[0, 1]$ such that

\[lim_{k \to \infty} u_{n_k} = x_0\]

Similarly, we also conclude that there exists a subsequence ${v_{n_k}}$ of ${v_{n}}$ and a 
point $x_0$ in $[0, 1]$ such that

\[lim_{k \to \infty} v_{n_k} = x_0\]

Knowing, however, that $f$ is continuous at $x_0$, we have 

\[f(u_{n_k}) = f(x_0) = f(v_{n_k})\]

for all $k$. Thus, we have

\[|f(u_{n_k}) - f(v_{n_k})| = 0\]

for all $k$.\\

This contradicts our assumption that there exists some $\epsilon > 0$ such that

\[|f(u_n) - f(v_n)| \geq \epsilon\]

for all $n$. Thus, we have proved that the function is uniformly continuous on the interval $[0, 1]$.\\

\textbf{c}\\

We want to prove that the function is not Lipschitz. We prove by contradiction


Suppose there $\exists C \in \mathbb{R}$ such that $|f(x) - f(y)| \leq C|x - y|$ for any $x, y \in [0, 1]$. 

\begin{align*}
    |\sqrt{x} - \sqrt{y}| &\leq C|x - y| = C|\sqrt{x} - \sqrt{y}||\sqrt{x} + \sqrt{y}|\\
    1 &\leq C|\sqrt{x} + \sqrt{y}|\\
    \frac{1}{c} &\leq |\sqrt{x} + \sqrt{y}|
\end{align*}

For any $x, y \in [0, 1]$ where $x \neq y$. However, this cannot be true. For instance, let $y = 0$
and $x = \frac{1}{c + 1}$. Since $\frac{1}{c} > \frac{1}{c + 1}$, $\frac{1}{c} > \sqrt{\frac{1}{c + 1}}$. This is 
a contradiction to our equation above. Thus, we have proved that the function is not Lipschitz.\\

\question \textbf{10}\\
We prove both directions using contradiction. First, we assume that $f$ is uniformly continuous, but 
it does not satisfy the $\epsilon - \delta$ criterion. Then, there exists some $\epsilon > 0$
such that for any $\delta > 0$, there exists some $u, v \in D$ such that $|u - v| < \delta$ but 
$|f(u) - f(v)| \geq \epsilon$.\\

Let $\delta = \frac{1}{n}$. Then, there exists sequences $u_n, v_n \in D$ such that $|u_n - v_n| < \frac{1}{n}$
and $|f(u_n) - f(v_n)| \geq \epsilon$.\\

Taking the limit as $n \to \infty$, we have

\[lim_{n \to \infty} |u_n - v_n| = 0\]

Which implies

\[lim_{n \to \infty} |f(u_n) - f(v_n)| = 0\]

As well since $f$ is uniformly continuous. However, this is a contradiction to our assumption that
$|f(u_n) - f(v_n)| \geq \epsilon$! 

Next, we prove the other way. Let the $\epsilon - \delta$ criterion be met, but $f$ is not uniformly continuous. 
Then, there exists some $\epsilon > 0$ such that for any $\delta > 0$, there exists some $u, v \in D$ such that
$|u - v| < \delta$ and $|f(u) - f(v)| < \epsilon$.\\

Then, there exist some N such that for any $n \geq n$ ,there are sequences 
$u_n, v_n \in D$ such that $|u_n - v_n| < \delta$ and $|f(u_n) - f(v_n)| < \epsilon$.\\

Furthermore, we let $\lim_{n \to \infty} |u_n - v_n| = 0$.

Thus, knowing that both sequences converge to the same point, the image of our sequences can be written as
$|(f(u_n) - f(v_n)) - 0| < \epsilon$. Thus, 

\[lim_{n \to \infty} |f(u_n) - f(v_n)| = 0\].

\question \textbf{14} \textbf{a}\\

We define $m = f(1)$. We want to prove that given the property that $f$ has the property 

\[f(u + v) = f(u) + f(v)\]

for all $u, v \in \mathbb{R}$, we have

\[f(x) = mx\]

for all $x \in \mathbb{R}$.\\

We prove by induction. Let's first consider all Natural Numbers. We know that $f(1) = m$. 
We have the base case as

\[f(1) = m \times 1 = m\]

Assume this property holds for all $x = k$. Thus, 

\[f(k) = mk\]

We want to prove that this property holds for $x = k + 1$. We have

\begin{align*}
    f(k + 1) &= f(k) + f(1)\\
    &= mk + m\\
    &= (k + 1)m
\end{align*}

Thus, we have proved that the property holds for all $x \in \mathbb{N}$.\\

We now consider $f(0)$. We can re-write this as 

\begin{align*}
    f(0) &= f(1 - 1)\\
    &= f(1) - f(1)\\
    &= m - m\\
    &= 0
\end{align*}

We now consider negative x values. We have

\begin{align*}
    f(-x) &= f(1 - 1 - x)\\
    &= (f(1) - f(1)) - f(x)\\
    &= (m - m) - f(x)\\
    &= 0 - f(x)\\
    &= -f(x)\\
    &= -mx
\end{align*}

Now. we wish to prove this for any rational number that can be expressed $\frac{p}{q}$. Let 
$q$ be positive. So, we wish to prove 

\[f(\frac{p}{q}) = m\frac{p}{q}\]

\begin{align*}
    mp &= f(p)\\
    mp &= f(q\frac{p}{q})\\
    mp &= f(\frac{p}{q} + \frac{p}{q} + ... + \frac{p}{q}) \text{ where there are q number of $\frac{p}{q}$ terms}\\
    mp &= f(\frac{p}{q}) + f(\frac{p}{q}) + ... + f(\frac{p}{q})\\
    mp &= qf(\frac{p}{q})\\ 
    m\frac{p}{q} &= f(\frac{p}{q})
\end{align*}

Thus, we have proved that $f(x) = mx$ for all rational numbers $x$.\\

\textbf{b}

We wish to prove that $f(x) = mx$ for all $x$. We are given that $f$ is continuous and 
that $f(x) = mx$ for all rational numbers $x$. 

If $f$ is continuous, then there exists some sequence ${x_n}$  such that 
if $x_n \to x_0$, then $f(x_n) \to f(x_0)$.

Since $\mathbb{Q}$ is dense in $\mathbb{R}$, we can find a sequence ${x_n} \in \mathbb{Q}$ such that
$x_n \to x_0$.

Thus, the $\lim_{n \to \infty} f(x_n) = f(x_0)$. 

Thus, we have proved that $f(x) = mx$ for all $x$.\\

\question \textbf{4}

We want to prove the function does not satisfy the $\epsilon - \delta$ criterion at the point
$x_0 = \frac{3}{4}$. Indeed, take $\epsilon = 1.875$. There is no positive number $\delta$ having
the property such that 

\[-1.875 < f(x) < 1.875\]

Around the point $$\frac{3}{4}$$. This, the function does not satisfy the $\epsilon - \delta$ criterion.

\question \textbf{1} \textbf{a} \\

No, let $f$ be a function that is defined at every point except $x = 0$ and let $g$ be defined only
at $x = 0$. Then, nether function is continuous at every point, but their sum is continuous at every point.\\

\textbf{b} \\

Yes, if the function squared is continous at every point, then the function is continuous at every point 
becuase every number must have a positive square root.\\

\textbf{c} \\

No, if $g$ and $f + g$ are continuous, that does not necessarily mean that $f$ is continuous. Let 
$f$ be defined as a piecewise function with a break at $x = 0$. Then, $f$ is not continuous at $x = 0$,
but $f + g$ is continuous at $x = 0$ since the $g$ component would "fill in" the gap.\\

\textbf{d} \\

Yes, since $\mathbb{N}$ is dense in $\mathbb{R}$, we can find a sequence of rational numbers
that converges to $x_0$. Thus, the limit of $f(x_n)$ is $f(x_0)$.\\

\question \textbf{6}

For this function to be continous at a point, $x^2 = -x^2$ at that point. 
Thus, $x = 0$. However, there is no irrational number whose square is $0$. Thus, the function is not continuous at any point.\\

\end{questions}

\end{document}
