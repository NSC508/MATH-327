\documentclass[addpoints]{exam}
\usepackage[legalpaper]{geometry}
\usepackage{etoolbox}
\usepackage[utf8]{inputenc}
\usepackage{amsmath, amsfonts}
\usepackage{enumerate}
\usepackage{hyperref}
\usepackage{relsize}
\newcommand{\sech}{\operatorname{sech}}
\newcommand{\R}{\mathbb{R}}
\newcommand{\Z}{\mathbb{Z}}
\newcommand{\lt}{\left}
\newcommand{\rt}{\right}
\newcommand{\pd}[2]{\frac{\partial #1}{\partial #2}}
\newcommand{\pdd}[3][]{\frac{\partial^{#1} #2}{\partial #3^{#1}}}
\newcommand{\lam}{\lambda}
\newcommand{\ansSpace}{\vspace{70mm}}


\title{Math 327 Homework 7}
\author{Sathvik Chinta}
\date{November 29th, 2022}

\begin{document}

\maketitle

\begin{questions}
\question \textbf{3} \\

Let $n = k + 1$. Thus, we can express

\[\sum_{k = 1}^{\infty} \frac{1}{(k+1)[\ln(k+1)]^\alpha}\]

As 

\[\sum_{n = 2}^{\infty} \frac{1}{n[\ln(n)]^\alpha}\]

The sequence of $\ln(n)$ is monotonically increasing. Thus, we know that $\frac{1}{n\ln(n)}$ must then decrease. We also know that since $n \geq 2$, $\frac{1}{nlog(n)} \geq 0$. Since this sequence is bounded and monotonically decreasing, 
from the Monotone Convergence Theorem, we know that the sequence $\frac{1}{n\ln(n)}$ must thus converge. From theorem 3.27 in the textbook, we know that if $a_1 \geq a_2 \geq a_3... \geq 0$, then the series 
$\sum_{n = 1}^{\infty} a_n$ converges if and only if the series 

\[\sum_{k = 0}^{\infty} 2^k a_{2^k}\]

Converges. Thus, we can plug this into our equation to get

\begin{align*}
    &\sum_{k = 1}^{\infty} 2^k \frac{1}{2^k[\ln(2^k)]^\alpha} \\
    = &\sum_{k = 1}^{\infty} \frac{1}{(k \ln(2))^\alpha} \\
    = &\frac{1}{\ln(2)^\alpha} \sum_{k = 1}^{\infty} \frac{1}{k^\alpha} \\
\end{align*}

Since $\frac{1}{\ln(2)^\alpha}$ is just a value no matter what $\alpha$ is, convergence depends on 
$\sum_{k = 1}^{\infty} \frac{1}{k^\alpha}$. Using the p-series test, we know that this series only converges if
$\alpha > 1$. Thus, we know that the series $\sum_{k = 1}^{\infty} \frac{1}{(k+1)[\ln(k+1)]^\alpha}$ converges if and only if $\alpha > 1$.

\question \\

We are given that the series $\sum_{n = 1}^{\infty} a_n$ converges and that the sequence $\{b_n\}$ is monotonic and bounded. 
We wish to prove that $\sum_{n = 1}^{\infty} a_n b_n$ must also converge. 

By the Monotone Convergence Theorem, we know that $\{b_n\}$ must then converge. Thus, we can write

\[b_n \leq M\]

For some value M. Then we can write 

\[\sum_{n = 1}^{\infty} a_n b_n \leq \sum_{n = 1}^{\infty} a_n M = M \sum_{n = 1}^{\infty} a_n\]

Since $\sum_{n = 1}^{\infty} a_n$, we know that $M \sum_{n = 1}^{\infty} a_n$ must converge as well. 

Since we have an upper bound that converges, we know that the series $\sum_{n = 1}^{\infty} a_n b_n$ must also converge by the 
Comparison Test.

\question \\

\textbf{i} \\

We can write $\sum_{n = 1}^{\infty} a_n$ as 

\begin{align*}
    \sum_{n = 1}^{\infty} a_n &= \left(\frac{1}{3} + \frac{1}{27} + \frac{1}{243} + \dots \right) + \left(\frac{1}{4} + \frac{1}{16} + \frac{1}{64} + \dots \right) \\
    &= \left(\frac{1}{3} + \frac{1}{3}^3 + \frac{1}{3}^5 + \dots \right) + \left(\frac{1}{2}^2 + \frac{1}{2}^4 + \frac{1}{2}^6 + \dots \right) \\
    &= \sum_{n = 0}^{\infty} \frac{1}{3}^{2n+1} + \sum_{n = 0}^{\infty} \frac{1}{2}^{2n} \\
    &\text{Let $i = 2n+1$ and $j = 2n$} \\ 
    &= \sum_{i = 1}^{\infty} \frac{1}{3}^i + \sum_{j = 0}^{\infty} \frac{1}{2}^j \\
    &= \sum_{i = 0}^{\infty} \frac{1}{3}^i - \frac{1}{3}^0 + \sum_{j = 0}^{\infty} \frac{1}{2}^j \\
    &= \frac{1}{1 - \frac{1}{3}} - 1 + \frac{1}{1 - \frac{1}{2}} \\
    &= \frac{5}{2}
\end{align*}

Thus, we have proved the series $\sum_{n = 1}^{\infty} a_n$ converges to $\frac{5}{2}$ and is thus convergent. 

\textbf{ii} \\

Let $n$ be even. Then, we can express $\frac{a_{n+1}}{a_n}$ as 

\[\frac{3^{-n + 1}}{2^{-n}} = \frac{(\frac{2}{3})^n}{3}\]

This series is monotonically decreasing. The limit as this series appraoches infinity is 0. Thus, the infimum is 0. 
The supremum is thus the beginning of our series. The first even number that $n$ can be is 2. Thus, we can write the supremum of this 
as $\frac{(\frac{2}{3})^(2)}{3} = \frac{4}{27}$. 

Now, let $n$ be odd. Then, we can express $\frac{a_{n+1}}{a_n}$ as 

\[\frac{2^{-n + 1}}{3^{-n}} = \frac{(\frac{3}{2})^n}{2}\]

This series is monotonically increasing. The limit as this series appraoches infinity is infinity. Thus, the sequence is unbounded
and there is no supremum. The infimum will be the beginning of our series. The first odd number that $n$ can be is 1. Thus, we can write the infimum of this
as $\frac{(\frac{3}{2})^(1)}{2} = \frac{3}{4}$.

The overall infimum thus is 0 while the overall supremum is infinity. 

The ratio test gives us no information about whether the series is convergent or not since it gives different values for $l$.
Wheb $n$ is even, $l < 1$ so the series should converge by the ratio test. When $n$ is odd, $l > 1$ so the series should diverge by the ratio test.
Thus, we cannot use the ratio test to determine whether the series converges or not.

\textbf{iii} \\

Let $n$ be even. Then, we can write $(a_n)^{\frac{1}{n}}$ as 

\[\frac{1}{2^n}^{\frac{1}{n}} = \frac{1}{2}\]

Now, let $n$ be odd. Then, we can write $(a_n)^{\frac{1}{n}}$ as 

\[\frac{1}{3^n}^{\frac{1}{n}} = \frac{1}{3}\]

The infimum is thus $\frac{1}{3}$ while the supremum is $\frac{1}{2}$.

Both of these values are $< 1$, thus they imply the sequence is convergent by the root test. Thus, we can use the root test to decide
if its convergent. 

\question

\question 

\question 
\end{questions}

\end{document}
